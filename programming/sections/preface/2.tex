% !TEX root = /home/weznon/programming/0207Programming/ftc-guides/programming/main.tex



\documentclass[../main.tex]{subfiles}

\begin{document}
\newpage
%Begin the stuff!
%This version of the template has some placeholder stuff so you can see what it looks like
\part{Preface}

The goal of this programming guide is for the reader to have a somewhat okay ability to program for an FTC team. However, we are mere mortals, so there are some caveats. 
\par The first is that we assume reasonable knowledge of Java. You do not need to be an expert, but if you are completely new a lot of this might go over your head.

\par The second is that a lot of FTC programming, and FTC skills in general, come from experience. You can read as many guides as you want, but some knowledge you will need to actually sit down and program for (this is especially true for debugging)

\par Finally, our teams, in comparison to the upper echelons of FTC teams, did not do anything too special. Sure, we used sensors, encoders, a little bit of PID. But we never really explored the limits of PID, motion profiling, advanced control theory stuff. 

When you look at teams like Gluten Free, with their crazy 6 glyph auto in Relic Recovery, and their 4.5 auto cycle auton in Rover Ruckus, you see something that this guide won't be able to teach you. But to get to that level, you need the basics - which this guide can teach.

\end{document}
